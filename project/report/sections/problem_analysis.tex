
随着消费信贷的快速发展和金融机构之间的激烈竞争, 信用评分的重要性日趋明显, 越来越多的金融机构借助信用评分模型寻求更好的信用风险防范策略。因此, 在信息技术和建模技术与金融业务深度融合的背景下, 信用评分成为近年来备受关注的热门话题。为了保证信用评分的科学性、有效性, 人们开发了诸如传统统计分析、人工智能等建模技术。
\subsection{信用评分方法综述}
传统的信用评分方法包括判别分析法、线性回归、逻辑回归等等。基于数据挖掘也不断涌现。决策树(CART)方早最早由经济学家Breiman等在1984年提出并使用。Arminger认为, 决策树是用一种非参数方法来分析分类变量, 这些变量为函数的连续解释变量。在分类树中, 通过基于单个输入的函数在每个节点上分割记录来构建二叉树。Ripley (1994)\cite{ref1} 提出, 神经网络模型通常应用于信用评分问题, 是涉及线性组合的非线性嵌套序列的线性统计模型。\\

其他信用评分方法还包括支持向量机(support vector machine, SVM),贝叶斯网络 (Bayesian network)等。Vapnik等人于1995年首次将支持向量机应用于信用评分。近年来, 贝叶斯网络开始被信用评分领域广泛应用, 郭春香和李旭升 (2009) 应用朴素贝叶斯和树增强贝叶斯分类器评估信用评分模型的准确性, 同时将贝叶斯网络算法与神经网络模型进行对比, 发现贝叶斯网络在信用评分中具有较高的准确性\cite{ref2} 。丁东洋和周丽莉 (2010) 利用贝叶斯网络模型评估信用评分模型中与参数相关的不确定性, 由此可以解决缺少实际违约数据问题, 提高模型的准确率\cite{ref3} \\

\subsection{存在问题}
经过统计和分析,现有的可参照的风控模型往往存在这样一些问题:
\begin{itemize}
	\item 对于监督训练任务来说,在构建模型阶段使用不同的算法,往往并不会带来结果的显著改变。这意味着有监督学习的效果很大程度上依赖于特征提取。
	\item 特征提取工程极大依赖于经验和手工分析,耗时且偶然性大。
	\item 许多大数据风控系统普遍存在数据的真实性不高、有效性仍需通过市场来论证、数据收集和使用过程中面临着合法性问题等。
\end{itemize}



