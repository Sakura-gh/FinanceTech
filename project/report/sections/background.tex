\subsection{研究背景}
风险管理是金融的本质之一,而风控是所有金融业务的核心。近年来,伴随金融科技(FinTech)浪潮,国家密集出台相关文件,要求加大互联网交易风险防控力度,鼓励通过大数据分析、用户行为建模等手段建立和完善可疑交易监测模型。\\

风险控制算法能以科学的方法把风险模式数据化,提供了客观的风险度量,还能提高风险管理效率,节省人力成本。风控算法的发展大致经历了如下阶段:从仿照有经验的风险分析专家设计信用判断条件,到采用回归分析等统计技术,再到现阶段大量引入了大数据和复杂模型。\\

其中,大数据风控是指利用数据分析和模型进行风险评估,为金融行业和个人用户提供全方位的安全保障。常见的业务场景有信贷、支付、登录、注册、精准营销等。关于大数据风控的应用,主要有如下两方面:
\begin{itemize}
\item 信贷场景中为信贷企业预防贷前、贷后等场景的欺诈风险。对借款人的历史借贷、消费特征等行为进行分析,前置性判断用户的还款能力(经济实力)和还款意愿(道德风险),为信贷决策提供可参考依据。
\item 构建整体风控解决方案,提供全方位的大数据分析。帮助互联网信贷企业,特别是小微企业的客户,更好地利用大数据提升风控和获益水平,减少潜在的信用和资金损失。
\end{itemize}


\subsection{研究意义}
在市场经济中,为了使市场和社会发挥作用,个人和公司都需要获得信贷。在这个过程中,银行起着至关重要的作用,它们决定着谁可以获得融资,以什么条件获得资金,还可以作出中断投资决定。 \\

因此,在高吞吐量的现代金融行业中,银行往往需要借助大数据风控的手段评估用户可能的违约情况。 \\

其中,违约概率算法可以猜测违约的可能性,是银行用来在贷前确定是否应授予贷款的方法。经典的违约概率估计整体流程包括:数据预处理(异常值和缺失值处理),特征工程(特征衍生、特征提取和特征选择),构建模型,模型评估,评分卡建立等。因为实际业务中,是数据的质量并不是永远那么完美,在特征工程和构建模型等等环节中,通常我们会使用到深度学习、半监督学习、弱监督学习等等方法去辅助传统监督学习算法。\\

本文尝试进行了整套智能风控流程的构建。其中,本文聚焦于特征工程和构建模型两个环节。在特征工程中,本文尝试在传统的特征工程的基础上,引入神经网络,实现对特征进行降维,提出了全新的方法实现更好的特征提取。进一步,在构建模型环节中,本文在多种模型上验证了新算法的优势。