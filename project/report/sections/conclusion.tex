本文创新地提出了自编码降维模块,其提取的3D特征比起传统特征工程在大部分分类器上都能有更优的性能。一方面,该模块的使用能一定程度上提升auc值;另一方面,比起手工进行特征的提取,自编码降维显然具有更高的稳定性和可靠性。其中,基于自编码降维的方法+LightGBM模型在该数据集上的表现最好。\\

由此可见,从整体上而言,利用Auto-encoder神经网络进行特征转换和提取在大数据风控的场景中是比较适用的。\\

当然,尽管如此,算法优化的幅度并不是十分显著。在LightGBM, Random Forest等分类器上,预测结果的auc值只得到$1\%\sim3\%$的提升。这带给我们更多关于如何更好在大数据风控领域应用好自适应编码模型的思考。\\

首先,本文将降维获得的特征固定为三维。这意味着数据集的信息会遭受一定程度上的丢失。后续实验可以尝试更多的可能维数k,探究k与auc值的关系。在未来,还可以尝试把自编码模块的神经网络由DNN改进为CNN或者其他模型。\\

在本场景、本数据集中,自编码降维模块体现了一个尚佳的水平,但并不意味着对于所有的风控场景和数据集来说,它的性能都能够得到保持。另一方面,本文的自编码降维模块的对比仅局限于一个传统特征工程,无法说明它能始终优于传统特征工程的平均水平。本文采取的分类器也局限于LightGBM, Random Forest, Logistic Regression, Neural Network Classifier,无法代表将其应用于所有监督学习模型的结果。在未来,应该在更多的更复杂和全面的测试集上增加对比测试,并且需要在相同条件下分析各个算法的优势和弊端,整合算法亮点,探索出一种准确性、稳定性等等方面都表现优异的算法。\\

关于信用评分系统的实践在本篇报告中告一段落,但之后持续的优化和完善工作并不会停止。或许金融科技领域的知识和技能挖掘也如同此次的实践一样,等待研究者们用充足的热情、持久的耐心以及不懈的毅力去践行。
